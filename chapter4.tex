\section*{LGS und Ausgleichsprobleme}

\subsection*{Direkte Verfahren}

\begin{karte}{Direkte Verfahren}
    Gegeben eine reguläre Matrix \( A = (a_{i,j}) \in \R^{n\times n} \) sowie ein Vektor 
    \( b \in \R^n \). Wir suchen den eindeutigen Vektor \(x \in \R^n\) mit 
    \[ Ax = b \Leftrightarrow \sum_{j=1}^n a_{i,j} x_j = b_i. \]
    \textit{Direkte Verfahren} liefern \(x \) bei exakter Arithmetik nach 
    endlich vielen Rechenoperationen.
\end{karte}

\begin{karte}{Cramersche Regel}
    Die eindeutige Lösung von \( x \) ist gegeben durch 
    \[ x_j = \frac{ \det(A[j]) }{ \det(A) }, \]
    wobei \( A[j] = (a_1, \ldots, a_{j-1}, b, a_{j+1}, \ldots, a_n) \in \R^{n\times n} \) ist.
\end{karte}

\begin{karte}{Rückwärtssubstitution/Vorwärtssubstitution}
    Sei \( R \) eine obere Dreiecksmatrix, gesucht ist \(x\) in \( Rx = z \).\\
    Durch \textit{Rückwärtssubstitution} bekommen wir \(x\):
    \begin{align*}
        x_n &= z_n / r_{n,n}, \\
        x_{n-1} &= (z_{n-1} - r_{n-1,n} x_n) / r_{n-1,n-1}, \\
        \vdots \\
        x_i &= (z_i - r_{i,i+1} x_{i+1} - \cdots - r_{i,n} x_n) / r_{i,i} \\
        \vdots \\
        x_1 &= (z_1 - r_{1,2}x_2 - \cdots - r_{1,n}x_n) / r_{1,1}.
    \end{align*}
    Analog \textit{Vorwärtssubstitution} für eine untere Dreiecksmatrix \(L\).
\end{karte}

\subsection*{Die LR-Zerlegung}

\begin{karte}{Gaußsche Dreieckszerlegung/LR-Zerlegung}
    Sei \(A\) regulär. Gesucht sind Matrizen \( L, R \) mit 
    \(L\) untere Dreiecksmatrix mit \(1\) auf der Diagonalen und 
    \(R\) obere Dreiecksmatrix. 

    Führe Gauß-Algorithmus aus (ohne normierte Diagonale und ohne Zeilenvertauschungen und Multiplikationen), 
    wobei \(A^{(i)}\) die umgeformte Matrix \(A\) im \(i\)-ten Schritt ist.
    Es gilt \( A^{(i+1)} = L_i A^{(i)} \). 
    Am Ende des Gauß-Algorithmus' haben wir \( R = A^{(n)} \) 
    und \( L = L_1^{-1}\cdots L_{n-1}^{-1} \).
    Wenn wir durch den Gauß-Algorithmus eine \gqq{Buchführungsmatrix} erstellen, 
    so ist diese \( L^{-1} \) und man muss diese nur noch invertieren.

    Dies funktioniert nur, wenn im Laufe der Rechnungen die Pivot-Elemente nicht verschwinden.
\end{karte}

\begin{karte}{LGS lösen mit LR-Zerlegung}
    Zur Lösung eines LGS \( Ax = b \) genügen drei Schritte: 
    \begin{enumerate}
        \item \( A = LR \) LR-Zerlegung
        \item \( Lz = b \) Vorwärtssubstitution
        \item \( Rx = z \) Rückwärtssubstitution
    \end{enumerate}
    Dies geschieht in \( \mathcal{O}(n^3) \).
\end{karte}

\begin{karte}{Diagonaldominante Matrix}
    Eine Matrix \( A \in \R^{n\times n} \) heißt \textit{diagonaldominant}, falls für jede Zeile der 
    Betrag des Diagonalelements größer ist als die Betragssumme der Außerdiagonalelemente, d. h. 
    \[ \abs{a_{i,i}} > \sum_{\substack{j=1\\j\neq i}}^n \abs{a_{i,j}} \;\forall i \in [n]. \]
    Eine diagonaldominante Matrix besitzt eine LR-Zerlegung und ist regulär.
\end{karte}

\begin{karte}{LR-Zerlegung mit Spaltenpivotsuche}
    Führe Gauß-Algorithmus aus und vertausche im \(k\)-ten Schritt die \( k \)-te Zeile 
    mit einer Zeile \( j \geq k \), für die gilt 
    \[ \abs{ a_{j,k}^{(k)} } = \max_{k\leq i \leq n} \abs{a_{i,k}^{(k)}}. \]
    Führe \gqq{Permutationsbuchführungsmatrix} \( P \) während des Algorithmus und tausche die Einträge 
    unter der Diagonale in \( L^{-1} \).

    Der Gauß-Algorithmus mit Spaltenpivotsuche liefert für jede reguläre Matrix \( A \)
    eine Permutationsmatrix \(P\) sowie Dreiecksmatrizen \(L\) und \(R\), sodass 
    \[ PA = LR. \]
    Darüber hinaus sind die Elemente von \(L\) betragsmäßig kleiner als \(1\).
\end{karte}

\begin{karte}{Rückwärtsanalyse der LR-Zerlegung}
    Die LR-Zerlegung mit Spaltenpivotsuche in Gleitkomma-Arithmetik der Genauigkeit \( \varepsilon \)
    angewandt auf \( A \in \R^{n\times n} \) breche nicht vorzeitig ab und die Anwendung auf das 
    Gleichungssystem \( Ax = b \) liefere die numerische Lösung \( \tilde{x} \). Dann existiert ein 
    \( \widehat{A} \in \R^{n \times n} \) mit \( \widehat{A}\tilde{x} = b \) und 
    \[ \frac{ ||\widehat{A} - A ||_\infty }{ ||A||_\infty } \leq 2 n^3 \rho_n(A) \varepsilon + \mathrm{o}(\varepsilon) \text{ für } \varepsilon \rightarrow 0 \]
    mit 
    \[ \rho_n(A) = \frac{ \max_{i,j,k} \abs{ a_{i,j}^{(k)} } }{\max_{i,j} \abs{ a_{i,j} }} \geq 1, \]
    wobei \( a_{i,j}^{(k)} \) die Elemente der Matrizen \( A^{(k)} \) sind. Die Zahl \( \rho_n(A) \) heißt 
    Wachstumsfaktor von \(A\).

    Diagonaldominante Matrizen erfüllen \( \rho_n \leq 2 \).
\end{karte}

\begin{karte}{Bandmatrix}
    Die Bandlänge einer Matrix \(A \in \R^{N\times N} \) ist das kleinste \( m\in \N \), sodass \( A_{i,j} = 0 \) 
    für \( \abs{i-j} > m \). 
    Der Aufwand für LR-Zerlegung ohne Pivotisierung ist in \( \mathrm{O}(m^2N) \) 
    und wir erhalten für \( m = \mathrm{O}(1) \) einen schnellen Löser, also ein Lösungsverfahren in 
    \( \mathrm{O}(N) \) Schritten.
\end{karte}

\subsection*{Die Cholesky-Zerlegung}

\begin{karte}{Positiv definiit}
    Eine Matrix \( A\in\R^{n\times n} \) heißt \textit{positiv definit} (auch \(A > 0\)), 
    falls \( A \) symmetrisch ist und \( \langle Ax, x\rangle_2 = x^T Ax > 0 \) 
    für alle \( x\in\R^n \setminus \set{0} \) gilt.

    Eine symmetrische Matrix \( A \in \R^{n\times n} \) ist genau dann positiv definit, wenn alle 
    Hauptminoren positiv sind.

    Sei \(A > 0\). Dann gelten 
    \begin{enumerate}
        \item \(A\) ist regulär,
        \item \( a_{i,i} > 0 \) und \( \abs{ a_{i,k} } < \sqrt{a_{i,i} a_{k,k}} \leq \frac{1}{2} (a_{i,i} + a_{k,k}) \) für \( i\neq k \),
        \item \( \max_{i,j} \abs{a_{i,j}} = \max_{k} \abs{a_{k,k}} \).
    \end{enumerate}
\end{karte}

\begin{karte}{Cholesky-Zerlegung}
    Sei \( A > 0 \). Dann existiert eine untere Dreiecksmatrix \( L \) mit positiven Diagonaldominanten, sodass 
    \[ A = L L^T \]
    ist. Diese Faktorisierung heißt \textit{Cholesky-Zerlegung} von \(A\).
\end{karte}

\begin{karte}{Cholesky-Zerlegung Konstruktion}
    \begin{align*}
        a_{1,1} &= l_{1,1}^2 &\Longrightarrow l_{1,1} &= \sqrt{a_{1,1}} \\
        a_{2,1} &= l_{2,1} l_{1,1} &\Longrightarrow l_{2,1} &= a_{2,1} / l_{1,1} \\
        \vdots &&&\vdots \\
        a_{n,1} &= l_{n,1} l_{1,1} &\Longrightarrow l_{n,1} &= a_{n,1} / l_{1,1} \\
        a_{2,2} &= l_{2,1}^2 + l_{2,2}^2 &\Longrightarrow l_{2,2} &= \sqrt{a_{2,2} - l_{2,1}^2} \\
        a_{3,2} &= l_{3,1} l_{2,1} + l_{3,2} l_{2,2} &\Longrightarrow l_{3,2} &= (a_{3,2} - l_{3,1} l_{2,1})/l_{2,2} \\
        \vdots &&& \vdots \\
        a_{n,2} &= l_{n,1}l_{2,1} + l_{n,2}l_{2,2} &\Longrightarrow l_{n,2} &= (a_{n,2} - l_{n,1}l_{2,1})/l_{2,2} \\
        a_{3,3} &= l_{3,1}^2 + l_{3,2}^2 + l_{3,3}^2 &\Longrightarrow l_{3,3} &= \sqrt{a_{3,3} - l_{3,1}^2 - l_{3,2}^2} \\
        a_{4,3} &= l_{4,1}l_{3,1} + l_{4,2} l_{3,2} + l_{4,3} l_{3,3} &\Longrightarrow l_{4,3} &= (a_{4,3} - l_{4,1} l_{3,1} - l_{4,2} l_{3,2}) / l_{3,3} \\
        \vdots &&& \vdots
    \end{align*}
\end{karte}

\begin{karte}{Rückwärtsanalyse für Cholesky-Zerlegung}
    Sei \( A \in \R^{n\times n} \) positiv definit. Der Algorithmus für die Cholesky-Zerlegung in 
    Gleitkomma-Arithmetik mit Genauigkeit \( \varepsilon \) liefere den Faktor \( \tilde{L} \). 
    Dann gilt 
    \[ \frac{ ||\tilde{L}\tilde{L}^T - A||_2 }{ ||A||_2 } \leq 8 n (n+1)\varepsilon + \mathrm{o}(\varepsilon) \text{ für } \varepsilon \rightarrow 0. \]
    Die Cholesky-Zerlegung einer \( n\times n \)-Matrix benötigt ca. \( n^3/6 \) Multiplikationen 
    (zzgl. der Auswertung von \(n\) Quadratwurzeln). Somit ist sie stabil.
\end{karte}

\subsection*{QR-Faktorisierung}

\begin{karte}{Householder-Reflexion}
    Eine \( m\times m \)-Matrix der Form 
    \[ H(v) := I_m - 2 \frac{ v v^T }{ v^T v } = I_m - 2 \frac{v v^T}{||v||_2^2} \]
    heißt \textit{Householder-Reflexion}. Householder-Reflexionen sind orthogonal, 
    denn 
    \[ H(v)^T = H(v) \text{ und } H(v)H(v)^T = H(v)^2 = I_m. \]
    Die Householder-Reflexion beschreibt eine Spiegelung an \( v^\bot \), es gilt 
    \[ H(v)v = -v, \quad H(v)w = w \;\forall w\in \langle \set{v}\rangle^\bot. \]
\end{karte}

\begin{karte}{QR-Zerlegung mittels Householder-Reflexionen}
    Zu jeder Matrix \( A\in \R^{m\times n} \) mit \( m \geq n \) und 
    \( \rk(A) = n \) existiert eine QR-Zerlegung mit 
    unitärem \( Q \in \R^{m\times m} \) und injektiver oberer Dreiecksmatrix 
    \[ R = \begin{pmatrix}
        r_{1,1} & \cdots r_{1,n} \\
        & \ddots & \vdots \\
        &&r_{n,n} \\
        0 &\cdots &0
    \end{pmatrix} \in \R^{m\times n}, \]
    mit \( r_{i,i} \neq 0 \) für \( i = 1,\ldots, n \).
    Zu jeder anderen QR-Zerlegung \( A = \tilde{Q}\tilde{R} \) von \(A\) 
    existiert eine unitäre Diagonalmatrix \( S \) mit 
    \( R = S\tilde{R} \) und \( Q = \tilde{Q} S^H \).
    Der Algorithmus zur QR-Zerlegung mittels Householder-Reflexionen 
    benötigt circa doppelt soviele Multiplikationen wie eine \(LR\)-Faktorisierung.
\end{karte}

\begin{karte}{QR-Zerlegung durch Householder-Reflexionen Verfahren Teil 1}
    Schritt 1: Definiere \(Q_1 := H(a_1 - \alpha_1 e_1) \in \R^{m\times m} \) 
    mit \( \alpha_1 = - \sign(a_{1,1}) ||a_1||_2 \neq 0 \).
    Es folgt 
    \[ A^{(2)} := Q_1 A = \left(\begin{array}{c|c|c|c}
        \alpha_1 & & & \\
        0 & a_2^{(2)} & \cdots & a_n^{(2)} \\
        \vdots &&&
    \end{array} \right), \]
    wobei \( a_k^{(2)} := Q_1 a_k \).
\end{karte}

\begin{karte}{QR-Zerlegung durch Householder-Reflexionen Verfahren Teil 2}
    Schritt \(k\) (\( 2 \leq k \leq p := \min\set{m-1, n} \)): 
    Nach dem \( (k-1) \)-ten Schritt liegt die Matrix 
    \( A^{(k)} \) vor mit der Struktur 
    \[ A^{(k)} = \left( \begin{array}{ccccccc}
        \ast & \cdots & \cdots & \cdots & \cdots & \cdots & \ast \\
        0 & \ast & \cdots & \cdots & \cdots & \cdots & \ast \\
        \vdots & 0 & \ddots & \vdots & & & \vdots \\
        \vdots & \vdots & \ddots & \ast & \ast & \cdots & \ast \\
        \vdots & \vdots & & 0 & & & \\
        \vdots & \vdots & & \vdots & & \tilde{A}^{(k)} & \\
        0 & 0 & \cdots & 0 & & &
    \end{array}\right), \]
    wobei \( \tilde{A}^{(k)} = (\tilde{a}_1^{(k)}, \ldots, \tilde{a}_{n-k+1}^{(k)}) \) 
    eine injektive \( (m-k+1)\times (n-k+1) \)-Matrix ist.
\end{karte}

\begin{karte}{QR-Zerlegung durch Householder-Reflexionen Verfahren Teil 3}
    Wir definieren die Householder-Reflexion \( Q_k := \left( \begin{array}{c|c}
        I_{k-1} & 0 \\
        \hline 
        0 & H(\tilde{a}_1^{(k)} - \alpha_k e_1)
    \end{array} \right) \), 
    \(\alpha_k = - \sign(\tilde{a}_{1,1}^{(k)}) ||\tilde{a}_1^{(k)}||_2 \), 
    die die erste Spalte von \( \tilde{A}^{(k)} \) auf ein Vielfaches von 
    \( e_1 \in \R^{m-k+1} \) spiegelt, die restlichen Spalten von \( \tilde{A}^{(k)} \) 
    transformiert, den Rest von \( A^{(k)} \) unverändert lässt.
    Damit ist die komplette orthogonale Transformation von \( A \) auf eine obere Dreiecksmatrix 
    \(R\) beschrieben. Wir haben 
    \[ R = Q_p Q_{p-1} \cdots Q_1 A = A^{(p+1)}, \quad Q := Q_1^T \cdots Q_p^T, \quad A = QR. \]
\end{karte}

\begin{karte}{Givens-Rotation}
    Eine \(m \times m\)-Matrix der Form 
    \[ G(l,k) = \left( \begin{array}{ccccccccccc}
        1 &&&&&&&&&& \\
        &\ddots&&&&&&&&& \\
        && 1 &&&&&&&& \\
        &&& c &&&& s &&& \\
        &&&& 1 &&&&&& \\
        &&&&& \ddots &&&&& \\
        &&&&&& 1 &&&& \\
        &&& -s &&&& c &&& \\
        &&&&&&&& 1 &&\\
        &&&&&&&&& \ddots & \\
        &&&&&&&&&& 1
    \end{array} \right) \]
    mit \( c^2 + s^2 = 1 \) und \( l < k\) heißt \textit{Givens-Rotation}. Für \( S \neq 0 \) 
    unterscheidet sich eine Givens-Rotation \( G(l,k) \) in genau \( 4 \) Einträgen von der 
    Einheitsmatrix \(I_m\). \( G_{l,l} = G_{k,k} = c \), \( G_{l,k} = -G_{k,l} = s \).
    \( G \) ist orthogonal: \( G^T G = I_m \)
\end{karte}

\begin{karte}{Givens-Rotation Verfahren}
    Es gibt ein \( \varphi \in (0,2\pi) \), sodass \( c = \cos(\varphi), s = \sin(\varphi) \) ist. 
    \( G(l,k) \) beschreibt also eine Rotation um den Winkel \( \varphi \) in der Ebene 
    \( \langle \set{e_l, e_k } \rangle \).
    Wähle \( c \) und \(s\) so, dass die \(k\)-te Komponente von \(x\) zu null wird. Falls \( x_l^2 + x_k^2 \neq 0 \), 
    so erreichen wir 
    \[ \begin{pmatrix}
        c & s \\ -s & c
    \end{pmatrix} \begin{pmatrix}
        x_l \\ x_k
    \end{pmatrix} \overset{!}{=} \begin{pmatrix}
        r \\ 0
    \end{pmatrix} \]
    durch die Setzungen 
    \[ r := \pm \sqrt{x_l^2 + x_k^2}, \quad c := \frac{x_l}{r}, \quad s := \frac{x_k}{r}. \]
    Mithilfe der Givens-Rotation kann man nun eine Matrix skuzessive in eine obere 
    Dreiecksgestalt bringen, indem man pro Spalte (von links nach rechts) immer 
    von unten nach oben die Einträge der Matrix eliminiert.
    Durch den letzten Schritt kommen wir auf \(R\) und \( Q = Q_1^T \cdots Q_p^T \), 
    wobei \(Q_i\) die Givens-Rotation aus dem \(i\)-ten Schritt ist.
\end{karte}

\begin{karte}{Überlaufvermeidung Givens-Rotation}
    Wenn \( x_l \) und \( x_k \) betragsmäßig sehr groß sind, kann man durch 
    \begin{align*}
        \abs{x_l} > \abs{x_k}: \quad &\tau := \frac{x_k}{x_l}, & c:= \frac{1}{\sqrt{1 + \tau^2}}, \quad &s := c\tau, \\
        \abs{x_l} \leq \abs{x_k}: \quad &\tau := \frac{x_l}{x_k}, & s := \frac{1}{\sqrt{1 + \tau^2}}, \quad &c := s\tau
    \end{align*}
    einen Überlauf verhindern.
\end{karte}

\begin{karte}{Givens-Rotation Aufwandsschätzung}
    Die Berechnung der QR-Faktorisierung einer vollbesetzten Matrix benötigt etwa 
    \( \frac{4}{3} n^3\) Multiplikationen für \( m \approx n \) und \( 2mn^2 \) 
    für \( m \gg n \).
\end{karte}

\subsection*{Lineare Ausgleichsprobleme}

\begin{karte}{Lineares Ausgleichsproblem}
    Durch den \textit{Gauß-Ausgleich} wird der Lösungsbegriff für 
    lineare Systeme \( Au \overset{!}{=} b \) sinnvoll verallgemeinert: 
    \[ \text{Finde } u^* \in \R^n: \quad ||Au^* - b||_2 = \min_{u\in\R^n} ||Au - b||_2. \]
    Das \textit{lineare Ausgleichsproblem} heißt \textit{überbestimmt}, falls \( m > n \) und 
    \textit{unterbestimmt}, falls \( m < n \).

    Hierbei ist äquivalent: 
    \begin{itemize}
        \item \( u^* \) löst das Ausgleichsproblem,
        \item \( u^* \) löst die Normalgleichung: \( A^T A u^* = A^T b \),
        \item \( u^* \) erfüllt \( A u^* = P_A b \), wobei \( \abb{P_A}{\R^m}{\R^m} \) 
        der Orthogonalprojektor (bez. \( \langle \cdot, \cdot\rangle \)) auf das Bild von \(A\) ist.
    \end{itemize}

    Das relative Residuum ist definiert durch 
    \[ \frac{ ||A u - b||_2 }{||b||_2} \]
\end{karte}

\begin{karte}{Differenz zweier Optimallösungen}
    Die Differenz \( u^* - v^* \) zweier Lösungen eines linearen Ausgleichsproblems 
    liegt in \( \ker(A) \): \( A(u^* - v^*) = 0 \). Das Ausgleichsproblem hat genau dann 
    eine eindeutige Lösung, wenn \( \ker(A) = \set{0} \). Dies kann im unterbestimmten 
    Fall nie und im überbestimmten Fall nur für \( n = \rk(A) \) eintreten.
\end{karte}

\begin{karte}{Minimum-Norm-Lösung}
    Unter allen Lösungen eines linearen Ausgleichsproblems nennen wir diejenige mit 
    minimaler euklidischer Norm die \textit{Minimum-Norm-Lösung} (MNL) \(u^+\).

    Dei MNL \(u^+\) ist die eindeutige Lösung der Normalgleichung in 
    \( \ker(A)^\bot \). Hat \(A\) Maximalrang, \( \rk(A) = \min\set{m,n} \), dann gilt 
    \[ u^+ = \begin{cases}
        (A^T A)^{-1} A^T b, & m \geq n, \\
        A^T (A A^T)^{-1} b, & m < n.
    \end{cases} \]
\end{karte}

\begin{karte}{Pseudoinverse einer Matrix und Kondition einer Matrix}
    Die Matrix \( A^+ \in \R^{n\times m} \), die jedem \( b\in \R^m \) die 
    MNL \(u^+\) des Ausgleichsproblems zuordnet, heißt 
    \textit{verallgemeinerte Inverse} oder \textit{Pseudoinverse} von \(A \in\R^{m\times n}\).
    Die Kondition von \(A\) ist definiert durch 
    \[ \kappa(A) := ||A|| \; ||A^+||. \]
    Ist \(A\) eine invertierbare quadratische Matrix, so gilt \( A^+ = A^{-1} \).
\end{karte}

\begin{karte}{Wedin (1973)}
    Es seien \(A, \Delta A \in \R^{m \times n}\) mit \( \rk(A) = \rk(A + \Delta A) \)
    und \( ||\Delta A||_2 \leq \delta ||A||_2 \), sowie \( b, \Delta b \in \R^m \) 
    mit \( ||\Delta b||_2 \leq \delta ||b||_2 \). Unter der Voraussetzung \( \kappa_2(A)\delta < 1 \) gilt 
    \[ \frac{ ||u^+ - v^+ ||_2 }{ ||u^+||_2 } \leq \frac{ \kappa_2(A)\delta }{ 1 - \kappa_2(A)\delta }
    \left( 2 + (\kappa_2(A) + 1)\frac{ ||Au^+ - b||_2 }{ ||A||_2||u^+||_2 } \right), \]
    mit der Notation \( u^+ = A^+ b \) und \( v^+ = (A + \Delta A)^+ (b + \Delta b) \). 
    Die Abschätzung ist annähernd scharf.
\end{karte}

\begin{karte}{Singulärwertzerlegung}
    Sei \( A \in \R^{m\times n} \) mit \(r = \rk(A) \leq \min\set{m,n}\). Dann existiert 
    eine Faktorisierung 
    \[ A = USV^T, \]
    wobei \( U \in \R^{m\times m} \), sowie \( V \in \R^{n\times n} \) orthogonale Matrizen 
    sind und \( S \in \R^{m\times n} \) eine Diagonalmatrix ist: 
    \[ S = \begin{pmatrix}
        \Sigma & 0 \\ 0 & 0
    \end{pmatrix}, \Sigma = \diag(\sigma_1, \ldots, \sigma_r) \in \R^{r\times r}, 
    \sigma_1 \geq \cdots \geq \sigma_r > 0. \]
    Darüber hinaus gelten \( A^+ = V S^+ U^T \) mit 
    \[ S^+ = \begin{pmatrix}
        \Sigma^+ & 0 \\ 0 & 0
    \end{pmatrix} \in \R^{n\times m}, \Sigma^+ = \diag(\frac{1}{\sigma_1}, \ldots, \frac{1}{\sigma_r}) \in \R^{r\times r}, \]
    sowie \( \kappa_2(A) = \frac{\sigma_1}{\sigma_r} \). Die Faktorisierung heißt 
    \textit{Singulärwertzerlegung} von \(A\). Die Spaltenvektoren \( u_i \) und \(v_j\)
    von \(U\) bzw. \(V\) heißen \textit{singuläre Vektoren} und die \( \sigma_k \) heißen 
    \textit{Singulärwerte} von \(A\). Es gilt 
    \[ Ax = \sum_{i=1}^r \sigma_i \langle x, v_i\rangle u_i \quad 
    A^+y = \sum_{j=1}^r \sigma_j^{-1}\langle y, u_j \rangle v_j. \]
\end{karte}

\begin{karte}{Berechnung der Singulärwertzerlegung}
    Seien \( \lambda_1, \ldots, \lambda_n \) die Eigenwerte von \( A^T A \)
    und \( \lambda_1 \geq \lambda_2 \geq \cdots \geq \lambda_r > 0 \) und 
    \( \lambda_{r+1} = \cdots = \lambda_n = 0 \). Es seien \( v_1,\ldots, v_n \) 
    die zugehörige ONB aus Eigenvektoren. Wir definieren 
    \[ \sigma_i := \sqrt{\lambda_i} \quad u_i := \frac{1}{\sigma_i} A v_i \;1\leq i \leq r. \]
    Seien \( u_j \) mit \( r < j \leq m \) weitere Orthonormalbasisvektoren.

    Dann bilden \( V = (v_1, \ldots, v_n) \), \(S\) wie im Satz definiert 
    und \( U = (u_1, \ldots, u_m) \) die Singulärwertzerlegung von \(A\).
\end{karte}